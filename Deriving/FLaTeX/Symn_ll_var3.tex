\resetsteps      % Reset all the commands to create a blank worksheet  

% Define the operation to be computed

\renewcommand{\operation}{  C \becomes A B + C $ where A is symmetric and stored in lower triangle $ }

\renewcommand{\routinename}{\operation}

% Step 1a: Precondition 

\renewcommand{\precondition}{
  C = \widehat{C}
}

% Step 1b: Postcondition 

\renewcommand{\postcondition}{ 
  C = AB + \widehat C
}

% Step 2: Invariant 
% Note: Right-hand side of equalities must be updated appropriately

\renewcommand{\invariant}{
  \FlaTwoByOne{A_{TL} B_T + C_T = \widehat C_T}
              {A_{BL} B_T + C_B = \widehat C_B} 
}

% Step 3: Loop-guard 

\renewcommand{\guard}{
  m( A_{TL} ) < m( A )
}

% Step 4: Initialize 

\renewcommand{\partitionings}{
  $
  A \rightarrow
  \FlaTwoByTwo{A_{TL}}{*}
              {A_{BL}}{A_{BR}}
  $
,
  $
  B \rightarrow
  \FlaTwoByOne{B_{T}}
              {B_{B}}
  $
,
  $
  C \rightarrow
  \FlaTwoByOne{C_{T}}
              {C_{B}}
  $
}

\renewcommand{\partitionsizes}{
$ A_{TL} $ is $ 0 \times 0 $,
$ B_T $ has $ 0 $ rows,
$ C_T $ has $ 0 $ rows
}

% Step 5a: Repartition the operands 

\renewcommand{\repartitionings}{
$  \FlaTwoByTwo{A_{TL}}{*}
              {A_{BL}}{A_{BR}}
  \rightarrow
  \FlaThreeByThreeBR{A_{00}}{*}{*}
                    {a_{10}^T}{\alpha_{11}}{*}
                    {A_{20}}{a_{21}}{A_{22}}
$,
$  \FlaTwoByOne{ B_T }
               { B_B }
\rightarrow
  \FlaThreeByOneB{B_0}
                 {b_1^T}
                 {B_2}
$
,
$  \FlaTwoByOne{ C_T }
               { C_B }
\rightarrow
  \FlaThreeByOneB{C_0}
                 {c_1^T}
                 {C_2}
$
}

\renewcommand{\repartitionsizes}{
  $ \alpha_{11} $ is $ 1 \times 1 $,
  $ b_1 $ has $ 1 $ row,
  $ c_1 $ has $ 1 $ row
}

% Step 5b: Move the double lines 

\renewcommand{\moveboundaries}{
$  \FlaTwoByTwo{A_{TL}}{*}
              {A_{BL}}{A_{BR}}
  \leftarrow
  \FlaThreeByThreeTL{A_{00}}{*}{*}
                    {a_{10}^T}{\alpha_{11}}{*}
                    {A_{20}}{a_{21}}{A_{22}}
$,
$  \FlaTwoByOne{ B_T }
               { B_B }
\leftarrow
  \FlaThreeByOneT{B_0}
                 {b_1^T}
                 {B_2}
$
,
$  \FlaTwoByOne{ C_T }
               { C_B }
\leftarrow
  \FlaThreeByOneT{C_0}
                 {c_1^T}
                 {C_2}
$
}

% Step 6: State after repartitioning
% Note: The below needs editing!!!

\renewcommand{\beforeupdate}{
  \FlaThreeByOneB{A_{00} B_0 + C_0 = \widehat C_0}
                 {a_{10}^T B_0 + c_1^T = \widehat c_1^T}
                 {A_{20} B_0 + C_2 = \widehat C_2}
}

% Step 7: State after moving of double lines
% Note: The below needs editing!!!

\renewcommand{\afterupdate}{
  \FlaThreeByOneT{A_{00} B_0 + a_{10} b_1^T + C_0 = \widehat C_0}
                 {a_{10}^T B_0 + \alpha_{11} b_1^T + c_1^T = \widehat c_1^T}
                 {A_{20} B_0 + a_{21} b_1^T + C_2 = \widehat C_2}
}

% Step 8: Insert the updates required to change the 
%         state from that given in Step 6 to that given in Step 7
% Note: The below needs editing!!!

\renewcommand{\update}{
$
  \begin{array}{l}
    \mbox{ $ C_0 = \widehat C_0 - a_{10} b_1^T $ } \\ 
    \mbox{ $ c_1^T = \widehat c_1^T - \alpha_{11} b_1^T $ } \\ 
    \mbox{ $ C_2 = \widehat C_2 - a_{21} b_1^T $ } \\ 
  \end{array}
$
}
